%===============================================================================

\chapter{Úvod}
\label{uvod}

\chapter{Technologie\ RFID}
\label{technologie_rfid}
RFID je zkratka pro {Radio-Frequency IDentification}, tedy identifikace pomocí rádiové frekvence. Tato technologie umožňuje bezdrátovou komunikaci na relativně krátkou vzdálenost\cite{The_RF_in_RFID}.
\par
RFID a bezdrátové technologie \cite{Smart_Cards_Tokens_Security}{ - str.321-347}

\section{Historie}
%\subsection{Čárové kódy}
Principy RFID byly poprvé použity v systému IFF (Identity: Friend or Foe) za 2. světové války Britskou armádou. Tento systém měl za úkol rozlišit vlastní letadla od nepřátelských. Proto byla vybavena nastavitelným {radio-majákem}, který byl schopen vysílat 6 identifikačních kódů. V padesátých letech minulého století se radio identifikace rozmohla z armády do celého letectví a používá se dodnes. %\cite{The_RF_in_RFID}\cite{Emulator_UHD_RFID_Tagu}.
\par
RFID vznikla jako alternativa k čárovým kódům. I když je výroba čárových kódů levnější, mají proti RFID mnoho nevýhod. Pro čtení musí čtecí zařízení přímo vidět na štítek s kódem. Nesmí být snížená jeho vizuální čitelnost, například špínou, popisovačem, nebo pokroucením. Zápis více informací se dá řešit pouze zvětšením plochy štítku, nebo použitím jemnějšího značení, které je ale viditelné z menší vzdálenosti. Modifikace dat uložených pomocí čárových kódů se dá prakticky řešit pouze tiskem nového kódu \cite{The_RF_in_RFID}\cite{Emulator_UHD_RFID_Tagu}.

\section{RFID\ tagy}
Primární využití RFID tagů spočívá v identifikaci objektů. Cena takových objektů je nesrovnatelně vyšší oproti ceně tagu. Pokud je značený objekt levný, tag musí být ještě levnější. Narozdíl od RFID čteček jsou tagy prakticky pořád v pohybu, ať už jako chytré karty, nebo identifikátory zboží, vlaků kontejnerů apod. Tagy tedy musí být velmi levné za vysoké odolnosti proti fyzickému poškození\cite{The_RF_in_RFID}.
RFID tag je systém skládající se minimálně z mikročipu, antény a pouzdra. Mikročip obsahuje pamět a logické obvody pro příjímání a odesílání dat čtecímu zařízení. Anténa přijímá signál z čtečky a poté jej zpětně rozptýlí (dále jen backscatter modulace) odesílanými daty. Pouzdro je potřeba pro udržení integrity tagu a proti vnějšímu poškození samotného čipu a antény\cite{RFID_explained}.
RFID tagy se dělí na aktivní, pasivní a částečně pasivní podle toho zda je jsou napájeny z externího, nebo interního zdroje\cite{The_RF_in_RFID}. Rozdíl je také v tom, kdo iniciuje komunikaci. Aktivní tag komunikaci zahajuje sám, zatímco komunikaci s pasivními tagy musí zahájit sama čtečka\cite{Hazardous_areas}. 
\par
Aktivni tagy jsou napájeny z vlastního zdroje, tedy baterie, která napájí nejenom přenos dat, ale i ostatní elektronické komponenty\cite{Survey_of_RFID_Tags}. Vývoj baterií ale postupuje velmi pomalu, vzhledem k technologii logických obvodů. Jedním z hlavních problémů návrhu těchto tagů je tedy zkrácení doby aktivity a snížení energie potřebné jak pro aktivní, tak pro klidové období tagu. Technologické skloubení těchto požadavků není jednoduché a výroba tagů se do jisté míry podobá výrobě běžných rádiových zařízení\cite{The_RF_in_RFID}. 
Částečně pasivní tagy obsahují baterii pouze k napájení pomocných komponent jako senzory nebo uživatelské rozhraní. Data jsou přenášena pomocí backscatter modulace, jako u pasivních tagů\cite{Survey_of_RFID_Tags}.
\par
%pasivní tagy


\section{RFID\ čtečky}

\section{Použití}

\chapter{Chytré karty Mifare Classic}
\label{chytre_karty_mifare_classic}

Historie chytrých karet \cite{Smart_card_handbook} {-} str. 37 ...
Největší rozvoj hlavně v Německu a Francii\cite{Smart_card_handbook} {-} str. 38-40
\par 
Co jsou vlastně chytré karty \cite{Smart_Cards_Tokens_Security} {-} str. 34 ... 
Použitá pamět EEPROM \cite{Smart_card_handbook} {- str. 39} ... 
Typy chytrých karet\cite{Smart_card_handbook}{ - str. 41-43} \cite{Smart_Cards_Tokens_Security}{ - str. 34-40}

\begin{itemize}
    \item paměťové
    \item mikroprocesorové
    \item bezdotykové
\end{itemize}

\par
Výroba Smart karet \cite{Smart_Cards_Tokens_Security}{ - str. 58-80} ??

\section{Varianty karet Mifare(?)}

\begin{itemize}
  \item MIFARE Classic 
  \item MIFARE Plus
  \item MIFARE Ultralight 
  \item MIFARE DESFire
\end{itemize}

\section{Známé zranitelnosti}

Smart Card Security \cite{Smart_Cards_Tokens_Security}{ - str. 224-254}

\chapter{Zařízení Chameleon Mini}
\label{zarizeni_chameleon_mini}







\chapter{Závěr}
\label{zaver}


%===============================================================================
