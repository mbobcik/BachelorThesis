%===============================================================================

\chapter{Úvod}
\label{uvod}

\chapter{Technologie\ RFID}
\label{technologie_rfid}
Zkratka RFID je akronym pro {Radio-Frequency IDentification}, tedy identifikace pomocí rádiové frekvence. RFID vznikla jako alternativa k čárovým kódům. I když je výroba čárových kódů levnější, mají proti RFID mnoho nevýhod. Pro čtení musí čtecí zařízení přímo vidět na štítek s kódem. Nesmí být snížená jeho vizuální čitelnost, například špínou, popisovačem, nebo pokroucením. Zápis více informací se dá řešit pouze zvětšením plochy štítku, nebo použitím jemnějšího značení, které je ale viditelné z menší vzdálenosti. Modifikace dat uložených pomocí čárových kódů se dá prakticky řešit pouze tiskem nového kódu.\cite{The_RF_in_RFID}























\chapter{Závěr}
\label{zaver}


%===============================================================================
